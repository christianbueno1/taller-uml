\documentclass[12pt,a4paper]{article}
\usepackage[spanish]{babel}
\usepackage[utf8]{inputenc}
\usepackage[T1]{fontenc}
\usepackage{geometry}
\geometry{margin=2.5cm}
\usepackage{titlesec}
\usepackage{enumitem}
\usepackage{hyperref}
\usepackage{graphicx}

\titleformat{\section}{\large\bfseries}{\thesection}{1em}{}

\begin{document}

\begin{center}
    \textbf{\Large ESCUELA SUPERIOR POLITÉCNICA DEL LITORAL}\\[0.2cm]
    \textbf{\large FACULTAD DE INGENIERÍA EN ELECTRICIDAD Y COMPUTACIÓN}\\[0.2cm]
    \textbf{\large INGENIERÍA DE SOFTWARE I}\\[0.5cm]
    \rule{16cm}{0.5pt}\\[0.5cm]
    {\Huge TALLER DE DISEÑO USANDO UML}\\[0.5cm]
    \rule{16cm}{0.5pt}\\[0.5cm]
    \textbf{Integrantes:}\\
    Alejandro Barrera\\
    Jose Murillo\\
    Christian Bueno\\[1cm]
\end{center}

\section*{Objetivo}
Familiarizarse con la representación y diseño de sistemas mediante el Lenguaje Unificado de Modelado (UML), aplicando distintos tipos de diagramas para documentar requerimientos, estructuras y comportamientos de un sistema.

\section*{Instrucciones}
\begin{enumerate}
    \item Forme grupos de 3 a 5 integrantes.
    \item Elija un sistema (por ejemplo, biblioteca, sistema de reservas, tienda en línea, etc.).
    \item Complete los ejercicios propuestos utilizando los diagramas UML adecuados.
    \item Presente y justifique sus modelos ante el grupo clase.
\end{enumerate}

\section{Descripción del sistema}
Describa brevemente el sistema elegido, sus funcionalidades principales y los actores que interactúan con él.

\vspace{2cm}

\section{Diagrama de Casos de Uso}
Elabore un diagrama de casos de uso que represente las funcionalidades principales del sistema y los actores involucrados. Incluya una breve descripción de cada caso de uso.

\vspace{7cm}

\section{Diagrama de Clases}
Diseñe un diagrama de clases que modele la estructura estática del sistema, identificando las clases principales, atributos y métodos, así como las relaciones entre ellas (herencia, asociación, composición, etc.).

\vspace{8cm}

\section{Diagrama de Secuencia}
Realice un diagrama de secuencia para uno de los casos de uso principales, mostrando la interacción entre objetos y el flujo de mensajes.

\vspace{8cm}

\section{Diagrama de Actividades}
Construya un diagrama de actividades que describa el flujo de trabajo de una funcionalidad clave del sistema.

\vspace{8cm}

\section*{Preguntas de Reflexión}
\begin{enumerate}[label=\arabic*.]
    \item ¿Qué ventajas ofrece el uso de UML en el desarrollo de sistemas?
    \item ¿Qué dificultades encontraron al modelar el sistema?
    \item ¿Qué tipo de diagrama les resultó más útil y por qué?
\end{enumerate}

\vspace{2cm}

\section*{Referencias}
\begin{itemize}
    \item \url{https://www.uml.org}
    \item \url{https://es.wikipedia.org/wiki/UML}
    \item Apunte de clase y recursos adicionales indicados por el docente.
\end{itemize}

\end{document}